\documentclass{scrartcl}
\usepackage{fullpage}
\title{GDC MapReduce Notes}
\author{Josh Snider}
\begin{document}
\maketitle
\subsection*{FP Review}
Data structures are not modified, new data structures are made instead.
This means we don't need to lock data structures. It also means that since
we don't have side effects, we make everything a separate thread and
put them together at the end.

Functional programming also allows you to use functions as arguments as the 
name implies. Functional programming is also good for generic yet typesafe
programming, but languages like OCaml can still screw this up.

Standard functional programming functions:
\begin{itemize}
\item Map - Apply a function to each element of a list and return a list made my
	concatenating the results together.
\item Fold - Track the result of calling f on each element and an accumulator,
	return the final accumulator. Can be either a left or right fold.
\end{itemize}
These are basically the Map and Reduce in MapReduce.
\subsection*{MapReduce}
Motivated by large scale data processing, parallelize across thousands of
computers, and make it easy for programmers to use. MapReduce has built-in
fault tolerance and network monitoring. The user basically has two functions
to provide
\begin{itemize}
\item map (in\_key, in\_value) -$>$ (out\_key, intermediate\_value) list -
	Input consists of records from various sources.
\item reduce (out\_key, intermediate\_value list) -$>$ out\_value list
\end{itemize}
After the map phase happens, we combine all the intermediate values for
a given key into one list and then start the reduce phase. It's good practice
for reduce to only have out\_value. We use barrier synchronization, to prevent
people from going to the reduce step while people are still mapping.

\subsubsection*{Optimization}
Each map function runs parallelly and the reduction for each key runs in parallel.
This means we have a bottleneck waiting for mapping to finish before we start
reduction. How do we optimize this?
\begin{itemize}
\item Slow-moving ``map'' tasks are run on multiple hosts and then we take the
	first one to finish.
\item Google considered making reducers lazy, but that has some design flaws
	that made it not used.
\item We can have ``combiners'' which run mini-reduce phases before the actual
	reduce in order to save bandwidth. If our reduce is both associative and
	commutative, then we can use it as our combine as well.
\end{itemize}

Written in C++, with Java and Python bindings. The dividing program tries
to divide up map tasks so that mappers are on the same computer as the data.
Tasks are chunked in 64MB which is the same size as the Google File System's
chunks.

\subsubsection*{Ensuring fault tolerance}
\begin{itemize}
\item We give up on tasks if certain key-value pairs reliably crash.
\item Mappers that fail before reporting results are redone.
\item Reducers don't claim to be done, until their results are reliably backed
	up.
\end{itemize}
\subsubsection*{Impact}
Has had great results at Google and is a shining jewel of functional programming.
Greatly simplifies large-scale computations. Lets programmer focus on problem
and let library handle messy details.

\subsection*{File Systems}
\end{document}
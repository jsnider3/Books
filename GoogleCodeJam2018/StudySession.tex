% !TEX TS-program = pdflatex
% !TEX encoding = UTF-8 Unicode

% This is a simple template for a LaTeX document using the "article" class.
% See "book", "report", "letter" for other types of document.

\documentclass[11pt]{article} % use larger type; default would be 10pt

\usepackage[utf8]{inputenc} % set input encoding (not needed with XeLaTeX)

%%% PAGE DIMENSIONS
\usepackage{geometry} % to change the page dimensions
\geometry{a4paper} % or letterpaper (US) or a5paper or....
% \geometry{margin=2in} % for example, change the margins to 2 inches all round
% \geometry{landscape} % set up the page for landscape
%   read geometry.pdf for detailed page layout information

\usepackage{graphicx} % support the \includegraphics command and options

% \usepackage[parfill]{parskip} % Activate to begin paragraphs with an empty line rather than an indent

%%% PACKAGES
\usepackage{booktabs} % for much better looking tables
\usepackage{array} % for better arrays (eg matrices) in maths
\usepackage{paralist} % very flexible & customisable lists (eg. enumerate/itemize, etc.)
\usepackage{verbatim} % adds environment for commenting out blocks of text & for better verbatim
\usepackage{subfig} % make it possible to include more than one captioned figure/table in a single float
\usepackage{fullpage}
\usepackage{hyperref}
% These packages are all incorporated in the memoir class to one degree or another...

%%% HEADERS & FOOTERS
\usepackage{fancyhdr} % This should be set AFTER setting up the page geometry
\pagestyle{fancy} % options: empty , plain , fancy
\renewcommand{\headrulewidth}{0pt} % customise the layout...
\lhead{}\chead{}\rhead{}
\lfoot{}\cfoot{\thepage}\rfoot{}

%%% SECTION TITLE APPEARANCE
\usepackage{sectsty}
\allsectionsfont{\sffamily\mdseries\upshape} % (See the fntguide.pdf for font help)
% (This matches ConTeXt defaults)

%%% ToC (table of contents) APPEARANCE
\usepackage[nottoc,notlof,notlot]{tocbibind} % Put the bibliography in the ToC
\usepackage[titles,subfigure]{tocloft} % Alter the style of the Table of Contents
\renewcommand{\cftsecfont}{\rmfamily\mdseries\upshape}
\renewcommand{\cftsecpagefont}{\rmfamily\mdseries\upshape} % No bold!

%%% END Article customizations

%%% The "real" document content comes below...

\title{Code Jam - Study Jam}
\author{Josh Snider}
\date{3/29/2018}

\begin{document}
\maketitle

\section*{Intro}

On 3/29/2018, \href{https://www.meetup.com/gdg-dc/}{Google Developer Group - Washington D.C.} decided to hold a study session to prepare people for the Google Code Jam. The event was cosponsored by the DC Android and Kotlin Washington DC User Group meetups. These are my notes from the event. The basic schedule of the event was discussing three problems from previous Code Jams. The Capital One Android team was most of the volunteers at the event.

\section*{About Code Jam}
The Code Jam schedule is \href{https://code.google.com/codejam/schedule}{here}, but essentially there's a qualification round the weekend of April 6th. Then three online rounds and a final onsite round. I've participated in these code jams before, I think I've made it to either the second or third round. Distributed Code Jam requires you to have made it to round three before, so I must have made it that far at least once. Points are awarded by speed and accuracy of your solutions. There's a cutoff for points to get to the next round. Code Jam has very few restrictions on languages. Today's practice problems can be found at goo.gl/fbnUH9

\section*{Problem 1}
Standing Ovation from the 2015 Qualification Round presented by Ahmad Ibrahim. You want everyone in the audience to clap, but every audience member has a shyness level that needs to be met before they will stand. What's the smallest number of people you can invite to get everyone to stand? I already had a solution from 2 years ago at https://github.com/jsnider3/Workspace/tree/master/Competitive/GoogleCodeJam15/Ovation. I made a new solution for this.

\section*{Problem 2}
Revenge of the Pancakes from 2016 Qualification Round presented by Rokas Leskevicius. We want to take a stack of pancakes and make sure they are all facing up. Our only power is to flip the top \textit{N} pancakes. How many times do you need to do this? You need to flip once for each time you have a `-' next to a `+' you then have to flip once more if that results in everything being face down.

\section*{Problem 3}
\href{https://code.google.com/codejam/contest/5304486/dashboard}{Alphabet Cake} from 2017 round 1A presented by Chuk-Yang Seng, we want to give each child a rectangular part of a cake covered in their initials. Cake can be divided unfairly, but must divide the entire cake. If there are multiple solutions, we can output any. I think we can just do a greedy solution. My solution was to just expand letters left, up, right, and then down as much as possible. That passed the small input.

\end{document}

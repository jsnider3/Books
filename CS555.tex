\documentclass{scrartcl}
\begin{document}
\title{CS 555 Final Exam Study Notes}
\author{Josh Snider}
\maketitle
\section*{Slide 7 Notes}
\begin{itemize}
\item What are the seven layers? (This was a midterm question, possibly also on the final)
\begin{itemize}
\item Application
\item Presentation
\item Session
\item Transport
\item Network
\item DLC
\item Physical
\end{itemize}
\item Types of switching
\begin{enumerate}
\item Message switching - entire message is sent as one giant block, sent from host to host to host. Kind of stupid.
\item Packet switching - each packet in the message is sent individually, real-time, reliable, prioritization
\item Circuit switching - dedicated circuit is created between beginning and end, phones.
\item Leased lines - permanently dedicated circuit, heavy constant traffic. See Spread Networks.
\end{enumerate}
\item Network functions
\begin{itemize}
\item Addressing - mapping names to addresses. translation done by nameservers.
\item Routing - selecting path for messages to take.
\item Congestion control
\end{itemize}
\item Types of Routing
\begin{enumerate}
\item Connection-oriented - establish a virtual circuit and send all data over it. Guarantees packets arrive in-order.
\item Connectionless - Like IP, each packet is routed seperately as datagrams. Has some issues that need to be handled.
\item Source routing - done as homework, not practical because it assumes everyone knows everything
\end{enumerate}
\item Ways to control congestion
\begin{enumerate}
\item Tell hosts to shut up.
\item Allocate more buffers.
\item Drop packets (connectionless only)
\item Require use of sliding window (connection-oriented only)
\item Reroute packets around congestion as if they were cars in traffic (connectionless only)
\end{enumerate}
\end{itemize}
\subsection*{Internetworking}
\begin{itemize}
\item Definition: routing among networks as opposed to within them, can be either connection-oriented or connectionless
\item Open Systems Interconnect - An open standard for connection-oriented internetworking. Expensive, supplanted by TCP/IP. Relies on X.25.
\item What is X.25?
\begin{itemize}
\item connection-oriented
\item requires X.21 bis and LAPB as layer 1/2 standards
\item Either over switched virtual circuit (SVC) or permanent virtual circuit (PVC).
\item Specifies interface between host and network switch. Interfaces between switches are left up to carrier.
\item X.25 Virtual Call: DTE sends CALL REQUEST packet, Receiver either replies with CALL ACCEPTED or CALL REJECTED. If accepted, communication begins. There's a sliding window which is usually 2 packets wide with packets of 256 bytes.
\item X.25 Fast Select: For short transmissions, can send 128 bytes with CALL ACCEPT/REJECT.
\item Connection Oriented Networking with X.75. X.75 defines interface between two X.25 networks. Makes bigger virtual network.
\end{itemize}
\item What is Asynchronous Transfer Mode?
\begin{itemize}
\item Connection-oriented packet-switched network
\item Sends cells of fixed 53 bytes, 5 bytes header, 48 bytes data.
\item Underlying technology for "Broadband Integrated Services Digital Network", which has its own stupid reference model.
\item Has associated hardware AND software. (sounds expensive!)
\item Viewpoints: Integrated access for users, network infrastructure for computers, backbone for lesser networks.
\item Reasons for small cell size
\begin{enumerate}
\item Reduced queueing delay
\item Minimize head-of-line blocking
\item Error correction for small cells and headers
\item Minimize jitter
\item Fixed format switching inefficiencies
\end{enumerate}
\item Routing is connection-oriented.
\item Basic element of routing is virtual channel. These are grouped into virtual paths.
\item Like X.25 we have PVCs and SVCs.
\item Connection setup is done with SETUP, CALL\_PROCEEDING, CONNECT, CONNECT\_ACK, RELEASE, and RELEASE\_COMPLETE messages.
\end{itemize}
\item SONET (Review)
\item Definition: Synchronous Optical NETwork, OC-1 is 51.84 Mbit/s, faster signals are made by multiplexing links.
\item Proprietary network protocols: Largely irrelevant, but Systes Network Architecture (SNA) was a popular IBM one and Internetwork Packet Exchange (IPX) was a Novell one.
\end{itemize}
\subsection*{The Routing Problem}
\begin{itemize}
\item Can either be done with shortest-path or "optimal" routing, based on whether you weight the links.
\item Primitive routing techniques include:
\begin{enumerate}
\item Source routing: Message contains list of nodes that must be visited on path to dest.
\item Static routing: predetermined paths that do not change.
\item Flooding: Spam everyone, then everyone spams everyone else. Horribly inefficient.
\end{enumerate}
\item Adaptive routing is the good alternative to static routing. Based on measuring network in action.
\item Shortest path spanning tree routing, either using Bellman-Ford or Dijkstra to compute the shortest path to each other node.
\end{itemize}
\end{document}
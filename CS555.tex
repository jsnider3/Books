\documentclass{scrartcl}
\usepackage{enumitem}
\begin{document}
\title{CS 555 Study Notes}
\author{Josh Snider}
\date{2015/05/11}
\maketitle
\section*{Seven layer model (top-to-bottom)}
\begin{itemize}
\item Application - getting shit done
\item Presentation - translating app data $\leftrightarrow$ network data  
\item Session - managing data connection
\item Transport - end-to-end connections and flow control
\item Network - routing and packet sending.
\item Data Link Control - framing and low-level error correction 
\item Physical - shooting photons and electrons at people!
\end{itemize}
\section*{Slide 7 Notes}
\begin{itemize}
\item Types of switching
\begin{enumerate}
\item Message switching - entire message is sent as one giant block, sent from
host to host to host. Kind of stupid.
\item Packet switching - each packet in the message is sent individually,
real-time, reliable, prioritization
\item Circuit switching - dedicated circuit is created between beginning and
end, phones.
\item Leased lines - permanently dedicated circuit, heavy constant traffic.
See Spread Networks.
\end{enumerate}
\item Network functions
\begin{itemize}
\item Addressing - mapping names to addresses. translation done by nameservers.
\item Routing - selecting path for messages to take.
\item Congestion control
\end{itemize}
\item Types of Routing
\begin{enumerate}
\item Connection-oriented - establish a virtual circuit and send all data over
it. Guarantees packets arrive in-order.
\item Connectionless - Like IP, each packet is routed seperately as datagrams.
 Has some issues that need to be handled.
\item Source routing - done as homework, not practical because it assumes
everyone knows everything
\end{enumerate}
\item Ways to control congestion
\begin{enumerate}
\item Tell hosts to shut up.
\item Allocate more buffers.
\item Drop packets (connectionless only)
\item Require use of sliding window (connection-oriented only)
\item Reroute packets around congestion as if they were cars in traffic
(connectionless only)
\end{enumerate}
\end{itemize}
\subsection*{Internetworking}
\begin{itemize}
\item Definition: routing among networks as opposed to within them, can be
either connection-oriented or connectionless
\item Open Systems Interconnect - An open standard for connection-oriented
internetworking. Expensive, supplanted by TCP/IP. Relies on X.25.
\item What is X.25?
\begin{itemize}
\item connection-oriented
\item requires X.21 bis and LAPB as layer 1/2 standards
\item Either over switched virtual circuit (SVC) or permanent virtual circuit
(PVC).
\item Specifies interface between host and network switch. Interfaces between
switches are left up to carrier.
\item X.25 Virtual Call: DTE sends CALL REQUEST packet, Receiver either replies
 with CALL ACCEPTED or CALL REJECTED. If accepted, communication begins.
 There's a sliding window which is usually 2 packets wide with packets of 256
 bytes.
\item X.25 Fast Select: For short transmissions, can send 128 bytes with CALL
 ACCEPT/REJECT.
\item Connection Oriented Networking with X.75. X.75 defines interface between
 two X.25 networks. Makes bigger virtual network.
\end{itemize}
\item What is Asynchronous Transfer Mode?
\begin{itemize}
\item Connection-oriented packet-switched network
\item Sends cells of fixed 53 bytes, 5 bytes header, 48 bytes data.
\item Underlying technology for "Broadband Integrated Services Digital Network",
which has its own stupid reference model.
\item Has associated hardware AND software. (sounds expensive!)
\item Viewpoints: Integrated access for users, network infrastructure for
 computers, backbone for lesser networks.
\item Reasons for small cell size
\begin{enumerate}
\item Reduced queueing delay
\item Minimize head-of-line blocking
\item Error correction for small cells and headers
\item Minimize jitter
\item Fixed format switching inefficiencies
\end{enumerate}
\item Routing is connection-oriented.
\item Basic element of routing is virtual channel. These are grouped into
virtual paths.
\item Like X.25 we have PVCs and SVCs.
\item Connection setup is done with SETUP, CALL\_PROCEEDING, CONNECT,
CONNECT\_ACK, RELEASE, and RELEASE\_COMPLETE messages.
\end{itemize}
\item SONET (Review)
\item Definition: Synchronous Optical NETwork, OC-1 is 51.84 Mbit/s, faster
signals are made by multiplexing links.
\item Proprietary network protocols: Largely irrelevant, but
System Network Architecture (SNA) was a popular IBM one and
Internetwork Packet Exchange (IPX) was a Novell one.
\end{itemize}
\subsection*{The Routing Problem}
\begin{itemize}
\item Can either be done with shortest-path or "optimal" routing,
based on whether you weight the links.
\item Primitive routing techniques include:
\begin{enumerate}
\item Source routing: Message contains list of nodes that must be visited
 on path to dest.
\item Static routing: predetermined paths that do not change.
\item Flooding: Spam everyone, then everyone spams everyone else. Horribly
inefficient.
\end{enumerate}
\item Adaptive routing is the good alternative to static routing. Based on
 measuring network in action.
\item Shortest path spanning tree routing, either using Bellman-Ford or
Dijkstra to compute the shortest path to each other node.
\end{itemize}
\section*{Slides 8 Notes}
\begin{itemize}
\item Metcalfe's Law: Value of a network to a user is proportional to $n^2$,
 where n is number of users.
\item Others argue that it's actually $n\log(n)$, these people have better
 evidence.
\item DARPA created ARPANET in 1968, which introduced RFC's.
\item In the 70's TCP/IP was developed as the protocol suite for ARPANET.
\item IETF now supports dozens of open protocols that make up the Internet
 Protocol Suite (IPS).
\item TCP/IP inventors were Vinton Cerf and Robert Kahn.
\item internets are made out of subnetworks.
\item A subnet contains bridges to relay frames within and routers to forward
 packets to other subnets.
\item MTU - Maximum Transfer Unit: Absolute maximum set by protocol, subnets
 can always make it smaller.
\end{itemize}
\subsection*{Internet Protocol (IP)}
\begin{itemize}
\item Connectionless, uses datagrams.
\item Delivers packets on a "best effort" basis with no guarantee of order.
\item Grown in popularity as a directly run protocol.
\item IPv4 uses 32-bit addresses. There were originally 4 classes, but that's
 historic now.
\item Uses a hierarchical domain name system.
\item Classless InterDomain Routing (CIDR) is a modern way of making blocks
 for the internet. I'm not entirely sure what it actually does?
\item Prefix notation, write blocks as 123.456/16, or 123.456.789/24.
\item IP Header contains a TIME TO LIVE (TTL) field which limits how many
 routers it can go through.
\item Address Resolution Protocol (ARP) - convert network address -> physical
 address.
\item Dynamic Host Configuration Protocol (DHCP) - runtime assignment of IP
 addresses.
\item IP Control Message Protocol (ICMP) - Passes control data between routers.
 Sends error statistics, pings, congestion control, klok times, routings loops
 etc.
\item ping and traceroute are ICMP utilities.
\item Fragmentation - algorithm for breaking up packets in cas MTU is smaller.
\begin{enumerate}
\item Divide data based on MTU.
\item Copy original IP header to IP header of each fragment.
\item Mark fragments specially.
\item Reassembled at destination using header flags.
\end{enumerate}
\item IPv6 (You should use it, IPv4 users are short-sighted fools)
\item Uses MTU discovery instead of IPv4 fragmentation
\item Has mandatory security (explain)
\item Slow adoption, even though IPv4 has run out of room.
\item INTERNET OF ALL TEH THINGZ!
\item IOT = embedded computers connected to internet (GENIUS!)
\item IPv4 doesn't have enough addresses, security is concerning.
\item Software defined networks is a research topic.
\item Gateway: Router that serves a subnet.
\end{itemize}
\subsection*{Security}
\begin{itemize}
\item Layer 2 link encryption, nodes using shared keys sending ciphertext
between them.
\item Layer 3 end-to-end encryption, body encrypted header not. Communities can
share keys apparently?
\item IP Security Protocol (IPSEC). Layer 3 protocol.
\item IPSEC has three primary mechanisms
\begin{enumerate}
\item IP Authentication Header for authentication/integrity.
\item IP Encapsulating Security Payload (ESP) for confidentially using
symmetrickey encryption.
\item Manual or automated key distribution - Internet key management protocols
(IKMP) being developed.
\end{enumerate}
\item AH and ESP have two modes:
\begin{enumerate}
\item Transport mode: host to host
\item Tunneling mode: firewall to firewall
\end{enumerate}
\item Firewall is another Layer 3 thing. It either filters things based on
application logic or based on packet headers.
\item Routing v forwarding: Routing determines where to send things. Forwarding
is the process of sending packets to where routing says it should go.
\item Routing either done by shortest-path (in hops) or collecting data and
trying to find an optimal solution.
\item Routing Information Protocol (RIP) uses a distance vector and UDP,
Unix builtin, messages exchange routing tables.
\item Open Shortest Path First (OSPF) use link state and IP, load balancing,
router exchange info with Link State Advertisement (LSA) messages.
\item OSPF design considerations
\begin{enumerate}
\item "Open"
\item Must support a variety of distance metrics.
\item Needs to respond dynamically to changes.
\item Should allow TOS (def?) routing
\item Should do load balancing
\item Security of routing updates
\end{enumerate}
\item OSPF supports three kinds of topologies (word choice?)
\begin{enumerate}
\item Point-to-point between two routers.
\item Multi-access network with broadcasting.
\item Multi-access networks without broadcasting.
\end{enumerate}
\item OSPF Protocols
\begin{enumerate}
\item Hello Protocol: Sends heartbeats.
\item Exchange Protocol: To synchronize routing databases.
\item Flooding Protocol: Distribute updates and ACK them.
\end{enumerate}
\item Border gateway Protocol
\begin{enumerate}
\item Internet gateways exchange routing info among admin domains.
\item gateway "advertises" that it can reach certain IP networks and its
distance to them.
\item Distance metrics not standardized
\item Exterior routing through autonomous systems (see below)
\item Uses TCP
\end{enumerate}
\item Autonomous systems: A region of the Internet under the control of a
single entity.
\end{itemize}
\section*{Slides 9 Notes}
\subsection*{Queueing theory}
\begin{itemize}
\item Definition: the mathematical foundation for
understanding/predicting the behavior of packet-switched networks.
\item One Model is a Single Server M/M/1 $\lambda$ is mean arrival rate and
$\mu$ is the mean service rate where $\lambda < \mu$.
\item Next slide confuses me. (Look at again.)
\item Exponential Distribution: PDF $\lambda e^{-\lambda t}$. CDF
$e^{-\lambda t}$
\item Formulas
\begin{itemize}
\item Utilization factor $\rho = \lambda/\mu$
\item Number in system $N = \rho / (1-\rho)$
\item Time in system (Little's formula) $T=N/\lambda$
\item Waiting time in queue $W=T-1/\lambda$
\item Number in queue $N_q = \lambda W$
\end{itemize}
\item Pollaczek-Khinchin Formula Total time in queueing node $T = \bar{x} +
\frac{\lambda * \bar{x}^2}{2(1-\rho)}$, where $\bar{x}$ = average service time,
$\lambda$ = average arrival rate, $\rho$ = utilization, and $x^2$ = second
moment of service (?).
\item WORK SOME EXAMPLES YOU LAZY FUCK!
\end{itemize}
\subsection*{Transport Protocols}
\begin{itemize}
\item Transport is the fourth layer of the seven layer model. It handles
end-to-end error and flow control, sometimes congestion as well.
\item Transport layers must provide reliable/consistent service in the face of:
\begin{enumerate}
\item Connectionless Networking
\item Circuit failures
\item Packet reordering
\end{enumerate}
\item Elements of transport protocols
\begin{enumerate}
\item Connection Management
\item Segment and reassemble application data
\item Recover from network failures
\item Error control and flow control
\item Multiplex multiple transport connections to one network connections.
\item Splitting and recombining: map one transport connection to multiple
network connections.
\end{enumerate}
\item There are two main internet transport protocols, User Datagram Protocol
(UDP) and Transmission Control Protocol (TCP). UDP is unreliable, TCP is the
opposite.
\end{itemize}
\subsubsection*{Stuff about TCP}
\begin{itemize}
\item Provides connection for connectionless IP.
\item Reliable and ordered.
\item Flow control via sliding windows
\item Uses Ports: In Berkeley socket = IP address + port.
\item Full-duplex transmission (Sends ACKs with responses)
\item Unique ID = (Sending IP/port, Receiving IP/port)
\item Delayed Duplicate Sequence Numbers: Very old packets from previous
connections may show up as if to appear in sequence for new connections.
How to prevent this? Either prevent sequence number reuse or purge old
packets from the network.
\item TCP uses a three-way handshake to setup connections and to release
connections.
\item TCP Startup, meet on pre-agreed port and then usually negotiate a
switch to a different one.
\item Uses slow-start to avoid congestion, start with minimum window size
of 1 MTU/64KB and every round trip we double the window size (unless
we see congestion). When we reach a threshold, we then start increasing
one MTU at a time instead of doubling. If we experience congestion, cut
the threshold in half and begin slow-start again.
\item TCP keeps a couple timers around, one for retransmission, one for
persistence, one for keep alive, and one for TIMED WAIT.
\item Retransmission timer needs to be dynamically adjusted due to
network differences. To do this we maintain an RTT variable to estimate the
round trip time. Each time we receive a non-timed out segment in time M,
we set $RTT \leftarrow \alpha RTT + (1+\alpha)M$, where $\alpha < 1$.
We also track a deviation estimator $D$ and update it as $D \leftarrow
\alpha D + (1 - \alpha)|RTT - M|$. We set the timeout to RTT + 4D.
\item What if we retransmit a segment and then get an ACK for it? Using
Karn's algorithm we don't update the RTT, but we double the timeout.
\item What if we need to send data without a full segment? Use the "push"
feature. This might be inefficient, so Nagle's algorithm accumulates data
in a buffer while waiting for an ACK and then sends whatever it has when
we get ACK'd.
\item TCP kinda sucks for high-performance networks due to knee-jerk
congestion control which cuts performance in half whenever it loses
something.
\item Network Address Translation: Use an IPv4 address to "front" for
many hosts and provide them private IPs from reserved space.
Then use TCP port multiplexing to identify who the thing is actually
intended for. Doesn't work for peer-to-peer. Since these hosts don't
have public IPs, it's slightly more secure since outsiders can't just
randomly call them up and give them some viruses.
\item TODO Work Examples
\end{itemize}
\section*{Slides 10 Notes}
\subsection*{Multimedia networking}
\begin{itemize}
\item IP can be used for voice (VOIP), video, graphics etc.
\item Multipurpose Internet Mail Extensions (MIME) does not require real time
support, some others do.
\item Lack of resource reservation can prevent IP from providing good
Quality of Service
\item Two approaches to ensure QoS: Integrated Services and Differentiated
Services. Also ATM networks with Multiprotocol Label Switching (MPLS).
\item Asynchronous Multimedia email
\begin{itemize}
\item Using MIME.
\item Text, hyperlinks, audio, graphic image format (GIF), jpeg, and mpeg
stuff.\end{itemize}
\item JPEG = joint photographic experts group
\item MPEG = motion pictures experts group
\item CERN created the WORLD WIDE WEB (WWW)!!!
\item HTTP = Hypertext Transfer Protocol
\item HTML = Hypertext Markup Language
\item Those two above do asynchronous multimedia.
\item Streaming can be used for pseudo-synchronous multimedia. Download
enough data that you can start playing it while the rest downloads without
worrying about interruptions.
\item Synchronous Multimedia is also possible with high-bandwidth connections.
Internet Multicast is capable of delivering real-time multimedia to multimedia
clients simultaneously. MBone also developed tools for multicast multimedia.
The main use case for synchronous multimedia is Skype and online lectures.
\item There are two important protocols for synchronous multimedia
\begin{enumerate}
\item Real-time transport protocol (RTP) - end-to-end networking for
real-time multimedia.
\item Session Initiation Protocol (SIP) - application-level protocol for
creating, modifying, and terminating sessions (i.e. skype calls)
\end{enumerate}
\item VOIP is usually cheaper than long-distance calls. Corporate customers
love it because they usually have excess internet and don't need to worry
about QoS issues. Can be done all-internet or mix internet and phone lines.
\end{itemize}
\subsection*{Multicasting}
\begin{itemize}
\item My wizard has a feat that allows multicasting as a full action.
\item Definition: sending a packet to a group of addresses in a network.
\item IPmc lets you send packets to a group address.
\item These "group" addresses are all in "class D" >= 224.0.0.0.
\item LANs can do this inherently.
\item WANs do a fancy tree.
\item Routers usually have to duplicate packets to get them to everyone.
Commercial routers can do this automatically.
\item Multicast needs its own routing protocol. Unicast is weak and for
cowards.
\item IPmc uses a many-to-many model of packet delivery
\begin{itemize}
\item Equivalent to full duplex
\item Anyone can send to anyone.
\item Useful for collaboration.
\item Appealing, but complex.
\end{itemize}
\item IPmc can also do Source Specific Multicast (SSM) which is one-to-many
\begin{itemize}
\item Equivalent to half-duplex
\item Only one host can spam everyone.
\item Useful for presentations.
\end{itemize}
\item IP Multicast Routing
\begin{itemize}
\item Internet Group Management Protocol (IGMP) runs on IP,
like "ARP for Multicast".
\item Distance Vector Multicast Routing Protocol (DVMRP): multicast
version of distance vector.
\item Multicast Open Shortest Path First (MOSPF): multicast version
of link state.
\item Multicast Border Gateway Protocol (MGBP): multicast version
of border gateway for interdomain routing.
\item (new) Protocol Independent Multicast (PIM) routing adds "relay points" formulticast-sparse portions of the internet. Also
available in sparse (PIM-SM), dense (PIM-DM), and SSM (PIM-SSM) flavors.
\item (proposed) Quality-of-Serivce Path First (QOSPF) would consider
availability along multicast paths.
\end{itemize}
\end{itemize}
\subsection*{QoS}
\begin{itemize}
\item Service parameters: data rate, delay, jitter, packet loss rate.
\item Quality-of-Service: best-effort, predictive, guaranteed performance.
\item QoS needs to be defined in context of capacity requirement.
\item Current internet is entirely is entirely best effort.
\item Integrated Services and Differentiated Services are trying to
change this. Still waiting on adoption by ISPs.
\item Internet Integrated Services: Group of proposed standards.
Controlled Load is supposed to offer service as if in lightly-loaded
best-effort situation. Guaranteed QoS guarantees a specific delay
and jitter provided a guarantee of traffic's peak rate.
\item Token Bucket: Improvement on "leaky bucket" rate control,
results in "rate-controlled" transmission.
\begin{enumerate}
\item Bucket starts with $b$ permits.
\item Permits arrive at rate per second $r$.
\item Every packet sent decrements bucket count.
\item No permit, no transmission.
\end{enumerate}
\item Bursty traffic: Applications may require periodic transmission
of large chunks of data. This correlation can destroy statistical models.
\item Bursty traffic is why TCP congestion is so important.
\item RSVP Messages (RFC 2205) - setup protocol for QoS to implement
IntServ Token Bucket.
\item Differentiated Services - Attempt to work around overhead of RSVP,
by making two classes of traffic one best-effort and large and one small and
QoS-sensitive. QoS is processed first and better. This is part of the net
neutrality controversy.
\end{itemize}
\subsection*{ATM Multicast}
\begin{itemize}
\item Reminder: ATM = Asynchronous Transfer Mode (what chapter is this from?
I should probably review it.)
\item ATM has built-in QoS
\item Services Categories: Constant Bit Rate (CBR), Variable bit rate:
real-time(RT-VBR), variable bit-rate: non real-time, (NRT-VBR), Available Bit
Rate (ABR)
Unspecified Bit Rate (UBR). Usage examples are VOIP, skype, email, web-surfing,
and ftp.
\item How to provide LAN service over ATM? Provide a central server and provide
VCs between all hosts using a LAN emulation server. This is called LAN
Emulation(LANE).
\item IETF does it using a server called ATM-ARP.
\item This isn't particularly common.
\item MPLS (defined above) can get ATM and IP to work together to make
good networks. Routers use ATM "pipes" for IP flows. These pipes have "labels"
that can be used for ATM switching. This is a "traffic engineering" thing.
\end{itemize}
\subsection*{Multicasting Research}
\begin{itemize}
\item NO OFF-THE-SHELF SOLUTIONS!
\item Distributed Virtual Simulation (DVS) (??)
\item Selectively Reliable Multicast Protocol (SRMP) - Reliable Multicast for
DVS. Combined RM of rarely-changing data with best-effort transmission of
frequently changing data. Has it's own stack. IETF is currently looking on a
TCP-friendly SRMP.
\end{itemize}
\subsection*{TCP Throughput Equation}
\begin{itemize}
\item $B = \frac{S}{RTT \times [\sqrt{\frac{2p}{3}} +
(12\sqrt{\frac{3p}{8}})p(1+32p^2)]}$,
where $S =$ Mean packet size, $p =$ packet loss event rate.
\end{itemize}
\section*{Security Slides}
\begin{itemize}
\item Basics of network security (review)
\begin{itemize}
\item Confidentiality (encryption/decryption)
\item Integrity (message authenticatication codes / message hashes)
\item Authentication (public/private keys with web of trust)
\item Access Control
\item Nonrepudiation (message hashes using private keys)
\item Key Distribution (using key exchange algorithms)
\end{itemize}
\item Secure Hash Functions, need to be hard to make a message given a hash
and to find a message with the same hash as a given message. These things
commonly turn out to be weaker than originally thought.
\item Three ways to try breaking an encryption scheme: Ciphertext only,
known plaintext, chosen plaintext (where Malory can encrypt message at will).
\item The hardness of breaking an encryption standard is proportional
to how likely it is to be used for important things.
\item Data Encryption Standard (DES) - Uses iterations to take a 64-bit
plaintext and 56-bit key and encrypt it (WEAK). Modified versions include
Advanced Encryption Standard (AES) and International Data Encryption
Algorithm (IDEA).
\item Public Key Cryptography uses seperate keys for encryption and decryption.
It's more secure and lets you identify people.
\item Public Key algorithms are based on problems which are much harder to
invert
\begin{itemize}
\item RSA algorithm is based on factoring the product of two large primes.
\item Diffie-Hellman is based on computing x given $y=a^{x}\%p$.
\end{itemize}
\item Public Key Infrastructure (PKI) - distributing web of trust basically, if
the internet had integrated PKI spam could always be traced back to its source
and we could then claim the skulls of spammers for Khorne.
\item Secure SHell (SSH) - is used to create secure terminal sessions over TCP,
it works well (depending on connectivity issues). Uses Diffie-Hellman key
agreement.
\item Secure Sockets Layer (SSL) - used to secure client-server sessions (https)
only occasionally has game-breaking Heartbleed bugs.
\item Message-Digest 5 (MD5) - fancy pants checksum
\item Simple Network Management Protocol (SNMP) - protocol to make
network management easy/consistent. Each managed object (that is network
attached device) has an SNMP agent and a Management Information Base (MIB).
MIB is basically a database of settings that can be edited. SNMP is mostly used
by telecom employees to monitor the network and then react (in)appropriately
as needed.
\end{itemize}
\section*{High-level protocols}
\begin{itemize}
\item Client-Server Model - server provides services and a number of clients
connect to it to take advantage of them by sending messages back and forth.
\item Some small things that use a client-server model are telnet, ssh, ftp, and
sftp. Two of these are security vulnerabilities and should not be used.
\item Simple Mail Transfer Protocol (SMTP) - transmit email as text,
mailservers talk to each other using TCP. Can either use Post Office Protocol
(POP3) or IMAP4 for clint/server. IMAP4 lets user scan headers for emails
before downloadingeth them.
\item X, list servs, and network news are other things that be.
\item Network Time Protocol (NTP) is a moderately important thing. It's used
to synchronize clocks on slave computers to a master's.
\item Social media loves Client-Server stuff. Facebook, twitter, youtube,
second life. These things drain your soul and poison the mind.
\item Remote Procedure Calls (RPC) are a legacyish way of distributed
communication that usually need reliable connections.
\item Network File Systems are also cool. NFS is the Unixy way, Google and
others have proprietary versions they made themselves.
\end{itemize}
\subsection*{World Wide Web protocols}
\begin{itemize}
\item Uniform Resource Locators (URLs) - they are what they is.
\item HyperText Transfer Protocol (HTTP), also in secure!
\item HyperText Markup Language (HTML)
\item Common Gateway Interface (CGI) - Generating webpages on the fly basically
nothing too interesting.
\item PHP - Programming language for server side ooga-booga.
\item Java Applets - Java for the client.
\item HTML5 - turing-complete modern alternative to the two above.
\item JavaScript - One more way to do client-side stuff.
\item XML - lesser version of JSON (DO NOT USE!)
\item Web Services - abuse a transaction-based web paradigm, also try to be
stateless. This has five required functions: self-description, publishing the
service descriptions, locating the service, establishing communications,
requesting initialization data, and exchange data with other web services.
\end{itemize}
\subsection*{Peer-to-peer Applications}
An increasingly interesting research avenue. Most likely going to be used for
things like Skype and whiteboard apps.
\section*{Formulas to know}
\subsection*{Comms}
\begin{itemize}
\item Bandwidth is the range of frequencies $W = f_h - f_l$ that can be sent
along an analog channel. Using it as jargon for digital capacity will be marked
wrong in this class.
\item K and M mean kilo and mega, in computers they are $2^{10}$ and $2^{20}$ in
communications they mean $10^3$ and $10^6$.
\item The Decibel is a measure of signal gain/loss it is equal to
$10\log_{10}(P_{meas}/P_{ref})$.
\item Bits per baud
\item Bytes, bits, and samples/sec
\item Shannon's law gives the theoretical max capacity (in bits per seconds)
of a channel. It is calculated as $C=B\log_2(1+\frac{S}{N})$, where $C$ is
capacity, $B$ is bandwidth, and $\frac{S}{N}$ is signal-to-noise ratio.
\item For a message of $b$ bits on a connection of $m$ bits / second, the
transmission time if $b/m$ seconds. If this message travels a distance $d$
meters at a velocity of $v$ m/s it will arrive with a propagation delay of $d/v$
seconds.
\item The Nyquist rate of an analog channel is equal to twice the highest
frequency in the signal. Analog signals need to be sampled at this rate in order
to minimize noise. The traditional way to represent analog signals as digital is
with N-bit pulse code modulation. N-bit PCM has a "noise of quantization"
$(S/N)_{PCM} = 2^{2N}$.
\item Common channels: switched voice (B channel) is 64 kbit/s, D channel is
16 kbit/s, T1 is 1.544 Mbit/s, SONET OC-1 is 51.84 Mbits/s, and OC-X is OC-1
* X.
\item With an n-bit divisor, CRC can detect all but $1/2^{n-1}$ errors, all
single-bit errors, all double-bit errors (with good divisor), all burst errors
up to (n-1) bits and most larger ones.
\end{itemize}
\subsection*{Networking}
\begin{itemize}
\item Efficiency and utilization
\item Timeout (S) - For TCP, $Timeout = RTT + 4D$, where $RTT$ and $D$ are
recalculated every time we receive a packet in time $M$ with the formulas
$RTT=\alpha RTT + (1-\alpha)M$ and $D=\alpha D+(1-\alpha)|RTT-M|$ with
$\alpha \in (0,1)$. It's related to the TCP window because increases in the $RTT$
trigger the slow-start procedure described a couple bullet points down to
restart.
\item Dijkstra routing - I've done this in 10 different classes by this point.
Algorithm is to work through the nodes in order of which are cheapest to get to
from the beginning, see if those open up faster routes to anywhere, and then
move on to the next node. When you're done, you have the fastest routes from
the source to everywhere.
\item Bellman-Ford routing - $O(n^3)$ breadth-first alternative to Dijkstra.
Given max number of hops $h$, we iterate over $x$ in $[1..h]$ setting $C[n][x]
\leftarrow min([C[j][x-1] + D[n][j]$ for $j$ in $G])$.
\item In a M/M/1 queue, with mean arrival rate $\lambda$ and mean service rate
$\mu$, the utilization factor $\rho =\lambda/\mu$, the number in the system
$N =\rho/(1-\rho)$, and the mean time in system (aka Little's Formula)
$T=N/\lambda$.
\item TCP window starts at 1 MTU (64 kb) and doubles as long as the RTT doesn't
go up, if it does we cut the threshold size in half and restart. If we get to
the threshold without experiencing congestion we start increasing the MTU by
one at a time until we reach the receiver max.
\end{itemize}
\section*{Vocab with definitions}
\begin{itemize}
\item Message switching - entire message is sent node-to-node until it reaches
the destination.
\item Packet switching - message is broken down into packets which are sent
independently and reassembled (or not) at the destination.
\item Circuit switching - single connection is made and used for all time.
\item Routing - Selecting a path for each packet to take through network.
\item Virtual Circuit - circuit simulated on a connectionless system
\item Virtual channel - see virtual circuit
\item shortest-path routing - routing using the shortest path in number of links
from source to destination.
\item Optimal routing - routing based on gathering global data and calculating
an optimal route from it (which is inherently inaccurate)
\item source routing - source gives list of nodes to go through which can't
be deviated from. (primitive)
\item Static routing - routes predetermined and unchanging. (also primitive)
\item Flooding - routing through no routing, just spam EVERYONE!
\item IETF - Internet Engineering Task Force, not superheroes, just the group 
that standardized TCP/IP and a couple related things.
\item RFC - request for comment, how IETF asked for comments on a standard.
\item IPS - internet protocol suite; TCP, IP, and some related things.
\item Subnet - Physical network nodes can directly communicate over.
\item Bridge - Connects nodes on same subnet.
\item Router - Connects nodes to different subnets.
\item MTU - Max transfer unit - largest packet/frame/etc. that layer can handle.
\item TTL - Time to live - max number of routers through which datagram can be
forwarded.
\item ARP - Address Resolution Protocol - Used to map IP to physical addresses
on LAN.
\item DHCP - Dynamic Host Configuration Protocol - Way to have server assign IP
addresses.
\item ICMP - IP Control Message Protocol - passes data between routers.
\item Fragmentation - Breaking a packet into smaller ones when the network can't
handle how big it is.
\item distance vector - uses hop count and UDP. Used by RIP routers.
\item link state - multiple metrics, used by Open Shortest Path First (OSPF).
\item BGP - Border Gateway Protocol - Lets ISPs trade routing info.
\item AS - Autonomous Systems - A region of the Internet controlled by a single
entity (i.e. an ISP). Have 1 bit identifier (ASN) and let internet scale.
\item inverse DNS - takes addresses and checks if they are valid.
\item NTP - Network Time Protocol
\end{itemize}
\textbf{ALL GLORY TO THE OMNISSIAH!}
\end{document}